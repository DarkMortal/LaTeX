\documentclass[14pt]{article}
\usepackage{amsmath}
\usepackage{amssymb}
\usepackage{cancel}
\usepackage{graphicx}
\usepackage{geometry}
\geometry{
	a4paper,
	total={170mm,257mm},
	left=20mm,
	top=17mm,
}
\begin{document}
	\title{Saptarshi's Math Notes}
	\maketitle
	\section{Inverse Hyperbolic Trigonometric Functions}
	\large{Pre-requisites/Synopsis :-}
	\begin{equation}
		\sinh(x)=\frac{e^x-e^{-x}}{2}
	\end{equation}
	\begin{equation}
		\cosh(x)=\frac{e^x+e^{-x}}{2}
	\end{equation}
	\begin{equation}
		\tanh(x)=\frac{e^x-e^{-x}}{e^x+e^{-x}}
	\end{equation}
	Let's assume $\sinh^{-1}(x)=\ln|f(x)|$ where $f(x)$ is some function of $x$ 
	\\ \begin{equation*}
	\therefore \frac{e^{\ln|f(x)|}-e^{-\ln|f(x)|}}{2} = x	
	\end{equation*}
	\begin{equation*}					%Equation environment with * won't be numbered
	\implies f(x)-\frac{1}{f(x)} = 2x		
	\end{equation*}
	\begin{equation*}
	\implies f^2(x)-2xf(x)-1 = 0		
	\end{equation*}
	Solving this quadratic equation, we get,
	\begin{equation*}
		f(x)=x\pm\sqrt{1+x^2}
	\end{equation*}
	If $f(x)=x-\sqrt{1+x^2}$, then $f(x)<0$ for all $x\in(-\infty,0)$
	\\ But $f(x)$ can't be -ve since $\ln|f(x)|$ will be an complex number.
	$\therefore f(x)=x+\sqrt{1+x^2}$
	\begin{equation}
		\therefore \boxed{\sinh^{-1}(x)=\ln|x+\sqrt{1+x^2}|}
	\end{equation}
	Similarly, we can prove,
	\begin{equation}
	\boxed{\cosh^{-1}(x)=\ln|x+\sqrt{x^2-1}|}
	\end{equation}
	Let $\tanh^{-1}(x)=\ln|f(x)|$
	\begin{equation*}
		\therefore \frac{e^{\ln|f(x)|}-e^{-\ln|f(x)|}}{e^{\ln|f(x)|}+e^{-\ln|f(x)|}}=x
	\end{equation*}
	\begin{equation*}
	\implies \frac{f(x)-\frac{1}{f(x)}}{f(x)+\frac{1}{f(x)}}=x
	\end{equation*}
	Using componendo-dividendo,
	\begin{equation*}
	\frac{1}{f^2(x)}=\frac{1-x}{1+x}
	\end{equation*}
	\begin{equation*}
	\implies f(x)=\sqrt{\frac{1+x}{1-x}}
	\end{equation*}
	\begin{equation}
	\therefore \boxed{\tanh^{-1}(x)=\ln(\sqrt{\frac{1+x}{1-x}})=\frac{1}{2}\ln(\frac{1+x}{1-x})}
	\end{equation}
	\section{Integration Problems}
	\begin{equation*}
		1. \int \frac{x^2}{(x\sin x+\cos x)^2}dx
	\end{equation*}
	\begin{equation*}
		=\int x \sec x \frac{x \cos x}{(x\sin x+\cos x)^2}dx
	\end{equation*}
	\begin{equation*}
		=x \sec x \int \frac{x \cos x}{(x\sin x+\cos x)^2}dx + \int \frac{\sec x+x\tan x}{x\sin x+\cos x}dx
	\end{equation*}
	\begin{equation*}
		=-\frac{x\sec x}{x\sin x+\cos x}+\int\frac{(\sec x+x\tan x)\times\cos^2 x}{(x\sin x+\cos x)\times\cos^2 x}dx+C_1
	\end{equation*}
	\begin{equation*}
		=-\frac{x\sec x}{x\sin x+\cos x}+\int\frac{\cancel{(x\sin x+\cos x)}}{\cancel{(x\sin x+\cos x)}\cos^2(x)}dx+C_1
	\end{equation*}
	\begin{equation*}
		\boxed{\tan x -\frac{x\sec x}{x\sin x+\cos x}+C}
	\end{equation*}
	\\ \begin{equation*}
		2.\int \frac{dx}{\sqrt[4]{(x-1)^3(x+2)^5}}
	\end{equation*}
	Let $a=x-1$
	\begin{equation*}
		\therefore I = \int a^{\frac{-3}{4}} (a+3)^{\frac{-5}{4}}da
	\end{equation*}
	\begin{equation*}
		\therefore I = \int a^{\frac{1}{4}} a^{-1} (a+3)^{\frac{-1}{4}} (a+3)^{-1}da
	\end{equation*}
	\begin{equation*}
		\therefore I = \int (\frac{a}{a+3})^{\frac{1}{4}} (a^2+3a)^{-1}da
	\end{equation*}
	\begin{equation*}
		\therefore I = \int e^{\frac{1}{4}\ln(\frac{a}{a+3})} (a^2+3a)^{-1}da
	\end{equation*}
	Let $u=\ln(\frac{a}{a+3})$
	\\ \begin{equation*}
		\therefore \frac{du}{da} = \frac{\cancel{a+3}}{a}\times\frac{(\cancel{a}+3)-\cancel{a}}{(a+3)^{\cancel{2}}}
	\end{equation*}
	\begin{equation*}
		du = 3(a^2+3a)^{-1}da
	\end{equation*}
	\begin{equation*}
		\therefore I = \frac{1}{3}\int e^{\frac{u}{4}} du
	\end{equation*}
	\begin{equation*}
		\implies I = \frac{4}{3} \int e^{\frac{u}{4}} d[\frac{u}{4}] = \frac{4}{3} e^{\frac{u}{4}} + C
	\end{equation*}
	Plugging in the substitutions we get,
	\begin{equation*}
		\therefore I = \frac{4}{3} (\frac{a}{a+3})^{\frac{1}{4}} +C=\boxed{\frac{4}{3} \sqrt[4]{\frac{x-1}{x+2}} +C}
	\end{equation*}
	Leibinz's Formula
	\begin{center}		%For center-alignment of any content
		\includegraphics[width=0.35\textwidth]{"./LeiBinz Formula.jpeg"} %This is an Image
	\end{center}
\end{document}